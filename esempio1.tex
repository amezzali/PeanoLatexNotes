\documentclass[12pt]{article}


\begin{document}
\title{Introduciamo \LaTeX\  al Peano}
\author{Andrea Mezzalira}
\renewcommand{\today}{23 Ottobre 2017}
\maketitle


\section {Plain Text}
Scrivete il vostro documento come volete, potete
fare delle righe lunghe o corte, sar\'a poi \LaTeX\  a sistemarle
per bene!

Le righe vuote terminano i paragrafi.


\section {Displayed Text}
Usate l'ambiente ``quote'' e ``quotation'' per aumentare il rientro del testo, utilissimo per evidenziare o citare del testo:

\begin{quotation}
Bla bla bla!

\em Ed ecco del testo in italic

\end{quotation}

\section{Liste}

\begin{enumerate}
\item 
L'ambiente ``enumerate'' crea delle liste numerate.
\end{enumerate}


Gli {\em item} in una lista possono contenere diversi paragrafi 

\begin{itemize}
  \item L'ambiente ``itemize'' crea liste con i  ``bullets'' come questo.
\end{itemize}

\noindent Infine, l'ambiente ``description'' vi permette di personalizzare le liste
\begin{description}
    \item[A] ad esempio con una lettera.
    \item[oppure con qualcosa di pi\'u lungo] in questo modo
\end{description}

\section{Tabelle}


Ecco una semplice tabella, inserita in un ambiente centrato per evidenziarla meglio

\begin{center}
Numbers of Computers on Earth Sciences Network, By Type.

\begin{tabular}{lr}
Macintosh&175\\
DOS/Windows PC&60\\
Unix Workstation or server&110\\
\end{tabular}
\end{center}


Di seguito invece una tabella pi\'u complessa, con i contorni:
\begin{center}
\begin{tabular}{|l|c|p{3.5in}|}
\hline
\multicolumn{3}{|c|}{Places to Go Backpacking}\\ \hline
Name&Driving Time&Notes\\
&(hours)&\\ \hline
Big Basin&1.5&Very nice overnight to Berry Creek Falls from
either Headquarters or ocean side.\\ \hline
Sunol&1&Technicolor green in the spring. Watch out for the cows.\\ \hline
Henry Coe&1.5&Large wilderness nearby suitable for multi-day treks.\\ \hline
\end{tabular}
\end{center}

\section{Formule ed equazioni}

Formule semplici come $x^y$ or $x_n = \sqrt{a + b}$ possono essere scritte direttamente lungo il testo inserendo la formula tra dollari.
Se volete utilizzare il simbolo dollaro ad esempio in questo modo \$2000,
dovete utilizzare il comando \verb+\$+.
\\

Per equazioni pi\'u complicate potete utilizzare l'ambiente  ``equation'' 
\begin{equation}
\left[
{\bf X} + {\rm a} \ \geq\ 
\underline{\hat a} \sum_i^N \lim_{x \rightarrow k} \delta C
\right]
\end{equation}
ma ci sono moltissime alternative!


\end{document}
